%This file discusses how to create and randomize a regular short- or long-answer question.

%The second file "m - question creation" assumes knowledge from this file, and contains only information related to multiple choice questions.

%%%%%%%%%%

%%%%% Index

% 0.0 - General setup

% 1.0 - Question formatting
%% 1.1 - Pre- and post- content
%% 1.2 - Writing questions and answers
%% 1.3 - Randomizing multiple varieties (and varieties with multiple parts)

% 2.0 - Randomization basics
%% 2.1 - Randomizing integers 
%% 2.2 - Randomizing from a set
%% 2.3 - Including arithmetic and variables in definitions
%% 2.4 - Randomizing and printing decimals
%% 2.5 - Dealing with floats and rounding

% 3.0 - More complicated randomization
%% 3.1 - Creating random numbers that must be different 
%% 3.2 - Selecting a set of values simultaneously
%% 3.3 - If/then commands
%% 3.4 - From a list of numbers with a specific value removed

% 4.0 - "Simplify" commands
%% 4.1 - Fractions
%% 4.2 - Square roots
%%% 4.2.1 - Dealing with square roots
%% 4.3 - Polynomials
%%% 4.3.1 - Advanced polynomial techniques
%% 4.4 - Numbers with commas



%%%%%%%%%%

%0.0 - General setup

The randomizer works by identifying text strings as bookends to definitions of questions, variables, solutions, and so forth. Anything written outside of these bookends is "safe" in the sense that it will never be found by the randomizer, such as this text.

This file is constructed so that it is able to run through the randomizer. Each section will create a question variant using the tools discussed. As such, bookend strings are introduced only in the variant, as other uses would break the file.

Question types are distinguished by file names. Anything starting with "q" must be short- or long-answer. Anything starting with "m" must be multiple choice. Suffixes "pre" and "post" are reserved for a long-answer question with multiple parts is split up over multiple files (the only place that this becomes relevant is in creating a grading guide file, as this may cause errors); pre for content before the questions and post for content after.



%%%%%%%%%%

%1.0 - Question formatting

%%1.1 - Pre- and post- content

For ease of use, content that has to be included before and after every question variation can be included separately with the pre and post bookends. When the randomizer chooses a question and begins its processes, it will automatically throw the pre content in before and the post content after. Note that this is only included in the assessment file and not in the associated solution file.

The randomizer will only look for one copy of each, so it cannot be used repeatedly.

This is a convenient way to write scores only once or make changes to spacing, both of which are which are convenient for editing. 

startpre \item {\bf (12 points)} endpre

startpost
\vfill 
endpost

Note that since everything between these bookends is brought in as a text string, too many newlines here and in the question can result in spacing issues in the created files. 

ie. \item {\bf (12 points)} 

Solve the following ... 



%%1.2 - Writing questions and answers

Each question variety must contain its own question and answer, and so these must be included in a larger grouping. These are the "all" bookends. The main block of python code automatically starts and ends the main enumeration LaTeX calls, so here everything must include an \item tag. No special consideration must be given to questions that are first or last in the exam. Since we have included the \item tag in the ''pre'' tag earlier in this case they are not needed  for the questions here, but are still needed with solutions since they do not receive the pre and post content.

Each "all" bookend must contain a "question" bookend pair and a  "solution" bookend pair. A "comments" bookend pair is optional, and can be used to distinguish how the question varieties actually differ from one another. These show up when creating a grading guide.

Content within the "question" bookends is sent to the exam file, content within the "solution" bookends is sent to the solutions file.

startall
startcomments
Used to describe variations. Does not appear on exams, only in grading guides.
endcomments

startquestion General question content.
endquestion

startsolution
\item Solution content.
endsolution
endall



%%1.3 - Writing multiple varieties (and varieties with multiple parts)

The randomizer randomly chooses one variety from the full list of content with the "all" bookends. Content in the "all" but outside of the other bookends is not used, but may potentially be seen by python, and therefore should be avoided (though it should never be treated as anything other than a text string).

Anything properly created with LaTeX should work as desired. Create varieties with multiple parts using enumeration.

Note that spacing may be designated between each part. Remember that any space created by the "post'' content will be included after the question content, and may throw off space for the last part.

startall
startcomments 
Question with multiple parts.
endcomments

startquestion Any preamble. \begin{enumerate}
\item A first part. \vspace{2cm}
\item A second part.
\end{enumerate}
endquestion

startsolution
\item \begin{enumerate}
\item First solution.
\item Second solution.
\end{enumerate}
endsolution
endall



%%%%%%%%%%

%2.0 - Randomization basics

The basic bookends for defining variables is "def". For example, the following string would create a variable 'a' with value 7.

startdef a = 7 enddef

It's important to understand how the randomization works to understand how to use it effectively. Everything to the left of the equals sign is treated as a text string (spaces are removed) and defines the variable. Everything to the right of the equals sign is evaluated by python, and a value is returned. In both the exam and solution files, a definition is created where the variable is assigned the value.

All variable assignments are done first, prior to choosing a question variety. If you look at the LaTeX output after running this file, before each question will be all the variable definitions. This occurs regardless of if the definition is used in the question.

Key to this working is the fact that if a definition is created twice in LaTeX, the most recent is used. This means that in a single question file is it important to not reuse variable names, as they will overwrite each other. In different question files this does not cause any problems.

Further, LaTeX does not allow numbers in definitions. Variable names must be only letters. 
(\x4 <- see)

Writing in LaTeX, the string \a calls the value that was assigned to "a" in Python.

startall
startquestion $a = \a$ endquestion

startsolution \item  endsolution
endall



%%2.1 - Randomizing integers 

To randomize from a string of consecutive integers, use the following.

startdef atwoone = np.random.randint(1,4) enddef

The way that python handles integers, this is a random number from the list {1,2,3}. Note that the first value is included, the last is not. 

Be careful; if the largest value is less than or equal to the smallest, then an error will occur in python.

startall
startquestion  I successfully chose then number $\atwoone$ at random. endquestion
startsolution \item  endsolution
endall



%%2.2 - Randomizing from a set

To randomize from a list of options, use the following.

startdef atwotwo = random.choice([-6,3,-2,5]) enddef

Any of these values can be chosen with equal probability. This is a great way to fix certain issues that may occur with random values in an interval, like cancellation. Lists of prime number or numbers of a specific parity can be used effectively.

startall
startquestion I successfully chose then number $\atwotwo$ at random. endquestion
startsolution \item  endsolution
endall



%%2.3 - Including arithmetic and variables in definitions

When defining variables, the python code assigns the value created to its own variable, with the same variable name. As a result, definitions can contain mathematical operations, as well as other variables so long as they are included in the question file in logical order.

startdef atwothree = np.random.randint(1,6) enddef
startdef btwothree = np.random.randint(6,10) enddef

startdef sumtwothree = atwothree + btwothree enddef
startdef diftwothree = atwothree - btwothree enddef
startdef bigtwothree = atwothree*100 enddef

Note the special notation python uses for exponents; a**b in python is the same as a^b everywhere else.
startdef powtwothree = btwothree**atwothree enddef 

Python uses the format math.log(value,base) = log_base(value).
startdef logtwothree = math.log(atwothree,btwothree) enddef

Multiple randomization calls can be included in the same definition.
startdef factortwothree = random.choice([2,3,5,7,11])*random.choice([13,17,19]) enddef

startall
startquestion \begin{enumerate}
\item $\atwothree + \btwothree = \sumtwothree$
\item $\atwothree - \btwothree = \diftwothree$
\item $\btwothree^{\atwothree} = \powtwothree$
\item $\atwothree \times 100 = \bigtwothree$
\item $\log_{\btwothree}(\atwothree) = \logtwothree$
\item $\factortwothree$ can be factored into two primes.
\end{enumerate} endquestion
startsolution \item  endsolution
endall

Conceptually, any mathematical operation that python is able to perform can be included here.

It is a good idea to always use curly braces around subscripts and exponents. If the defined value is one character long and you don't it doesn't cause problems, but if the length is more than one character only the first will be raised or lowered, the same as the LaTeX error for 2^13.



%%2.4 - Randomizing and printing decimals

There are obviously multiple ways to create and display decimals.

The following variables would then have two and three decimal places, respectively. In each case, values will be between one and two.

Divide creatively;
startdef atwofour = float(decimal.Decimal(np.random.randint(101, 200))/100) enddef
Randomize a float value and round to the desired number of decimal places;
startdef btwofour = round(random.uniform(1.001,2.0),3) enddef

startall
startquestion $\atwofour$, $\btwofour$ endquestion
startsolution \item  endsolution
endall

Because of how python handles float values, trailing zeros are NOT printed. That is, if a variable 2.10 is created, it will appear as 2.1 without proper formatting commands. In cases where a specific number of decimal places are required, like a problem involving money, use the format command.

startdef tooshorttwofour = 10.1 enddef
startdef moneytwofour = '{:.2f}'.format(tooshorttwofour) enddef
startdef longertwofour = '{:.5f}'.format(tooshorttwofour) enddef

startall
startquestion Two decimal places: $\moneytwofour$. Five: $\longertwofour$. endquestion
startsolution \item  endsolution
endall



%%2.5 - Dealing with floats and rounding

Python evaluation isn't perfect, and tends to fall of by the tenth decimal place. This is an issue with float values. Avoid using float values in intermediate steps whenever possible.

Use the round command to round to a specific number of decimal places.

startdef atwofive = round(0.123456,2) enddef
startdef btwofive = round(0.123456,5) enddef

startall
startquestion $\atwofive < \btwofive$ endquestion
startsolution \item  endsolution
endall



%%%%%%%%%%

%3.0 - More complicated randomization

%%3.1 - Creating random numbers that must be different 

You can use arithmetic to guarantee that two numbers will not be equal by randomizing the difference.

startdef athreeone = random.choice([4,5,6]) enddef
startdef bthreeone = athreeone + random.choice([-2,-1,1,2]) enddef

startall
startquestion $\athreeone \neq \bthreeone$ endquestion
startsolution \item  endsolution
endall

Other methods for doing this are included in Sections 3.2 and 3.4.



%%3.2 - Selecting a set of values simultaneously

If you need to create a number of variables that interact in specific ways, one option is to select random sets. The setup for doing so is as follows.

startdef setthreetwo = np.random.choice(['[2, 5, 6]','[3, 7, 7]','[12, 6, 9]']) enddef 

An example that this has been useful for is quadratics that require the quadratic formula to factor, but can't be simplified (for consistent grading).

This creates a variable, setthreetwo, that is one of [2,5,6], [3,7,7], or [12,6,9]. Correct formatting is necessary here. Then, define variables to be the values of the python list. Note that python lists always start at position 0.

startdef athreetwo = setthreetwo[0] enddef
startdef bthreetwo = setthreetwo[1] enddef
startdef cthreetwo = setthreetwo[2] enddef

startall
startquestion $\setthreetwo$ contains $\athreetwo,\bthreetwo$, and $\cthreetwo$. endquestion
startsolution \item  endsolution
endall



%%3.3 - If/then commands

LaTeX can use numeric if/then commands. This can be particularly useful in creating question files that require fewer varieties. 

Correct usage can be tricky, so be sure to test any questions using these carefully.

This can be particularly useful for displaying tricky functions correctly, or randomizing alternate functions.

startdef controlthreethree = random.choice([-8,-5,-4,4,5,8]) enddef enddef
startdef athreethree = random.choice([1,2,3,4]) enddef
startdef topthreethree = random.choice([0,1]) enddef

startall
startquestion \begin{enumerate}
\item If I add $\controlthreethree$ to $y=x^2$, the graph shifts \ifthenelse{\controlthreethree >0}{up}{down}.
\item $y = \ifthenelse{\athreethree = 1}{4^x}{\athreethree(4^x)}$
\item $y = \ifthenelse{\topthreethree = 1}{x}{\frac{1}{x}}$
\end{enumerate}
endquestion

startsolution \item  endsolution
endall

Note that "then" and "elses" can be left empty if you want nothing printed.



%%3.4 - From a list of numbers with a specific value removed

A separate python function had to be created to remove values from potential sets. (There were issues with using the ones already in existence due to how variables exists in pythons memory. This is a known issue with python, unlikely to be fixed anytime soon.)

Two workaround uses the command call set_remove. A set is created from items in the first, after removing items in the second. Note that items in the second that are not in the first do not cause errors.

This creates another way of choosing two different, non-equal values.

startdef athreefour = random.choice([3,4,5]) enddef
startdef bthreefour = random.choice(set_remove([1,2,3,4,5,6],[athreefour,6,7,8])) enddef
startdef listthreefour = set_remove([1,2,3,4,5],[athreefour]) enddef

startall
startquestion $\athreefour, \bthreefour, \listthreefour$ endquestion
startsolution \item  endsolution
endall

Inputs can be lists [], sets {}, or tuples (). Further, using Python language, range(1,4) = {1,2,3} can be a useful shorthand.



%%%%%%%%%%

%4.0 - "Simplify" commands

A number of specific Python functions were created to deal with specific Python issues. In some cases, the output is something that Python is incapable of understanding and causes crashes. To deal with these, the string "simplify" is used. If a function begins with this character string, Python will not attempt to understand the evaluation of that definition. This means that you cannot perform any additional operations in Python with any variables defined with a simplify command.

Usually, this means commands where the output is LaTeX code, which is unreadable to Python.



%%4.1 - Fractions

Simplify_fraction is used to write fractions in simplest form. It works correctly if the denominator is 1. 

The input for this command is simplify_fraction(numerator,denominator)

In the case of negative fractions, the numerator is written as a negative number. 

Note that the output is a \frac command. Use \displaystyle to have it display in larger size.

startdef afourone = random.choice([4,8,12,16]) enddef
startdef bfourone = random.choice([-2,4,-6,-8]) enddef

startdef fracfourone = simplify_fraction(afourone,bfourone) enddef

startall
startquestion $\dfrac{\afourone}{\bfourone} = \fracfourone$ or $\displaystyle \fracfourone$ 
endquestion
startsolution \item  endsolution
endall



%%4.2 - Square roots

Simplify_sqrt can be used to write simplified square roots. This input is what goes inside the square root.

Note that there are obvious workarounds in variable creation to using this particular code, but it is nice and simple to use.

It does NOT work with negative inputs. See Section 4.2.1 for an alternate workaround.

startdef rootfourtwo = random.choice([8,12,20]) enddef
startdef simplifiedfourtwo = simplify_sqrt(rootfourtwo) enddef

startall
startquestion  $\sqrt{\rootfourtwo} = \simplifiedfourtwo$ endquestion
startsolution \item  endsolution
endall


%%%4.2.1 - Dealing with square roots

In case the values of a simplified root are needed for further manipulation, the command sqrt_output can be used. Here, an input x returns a list [a,b] where \sqrt{x} = a\sqrt{b}. Like randomizing a set of lists, define variables to grab these outputs.

startdef sqrtlistfourtwo = sqrt_output(rootfourtwo) enddef
startdef outfourtwo = sqrtlistfourtwo[0] enddef
startdef infourtwo = sqrtlistfourtwo[1] enddef

startdef wowfourtwo = simplify_fraction(outfourtwo,2) enddef

startall
startquestion $\dfrac{1+\sqrt{\rootfourtwo}}{2} = \dfrac{1}{2} + \ifthenelse{\wowfourtwo = 1}{}{\wowfourtwo} \sqrt{\infourtwo}$
endquestion
startsolution \item endsolution
endall



%%4.3 - Polynomials

Simplify_polynomial correctly writes polynomials, given a variable and sequence of numeric coefficients. The list of coefficients must be given based on degree of terms. 

Note that variables can be complicated, but, for example, xyz would have larger exponents display incorrectly; xyz^3 instead of (xyz)^3. Use (xyz) instead.

startdef afourthree = random.choice([-5,0,5]) enddef
startdef nicethreefour = simplify_polynomial('x',[3,-1,afourthree]) enddef
startdef nastythreefour = simplify_polynomial('xyz',[1,0,0,afourthree]) enddef

startall
startquestion  $\nicethreefour = \nastythreefour$ endquestion
startsolution \item  endsolution
endall



%%%4.3.1 - Advanced polynomial techniques

(Credit to Avantha Kodithuwakku turning a bug into a feature.)

An issue with simplify_polynomial is that, if the leading coefficient is zero, the new first time will have a sign displayed. For example, simplify_polynomial('x',[0,1,1]) returns +x+1. This is actually incredibly useful when writing more complicated functions, such as multivariate functions or functions written in non-standard ways.

For example, writing ax+by where b might be negative can be done easily this way.

startdef cfourthree = random.choice([-4,4]) enddef
startdef dfourthree = random.choice([-10,-9,9,10]) enddef

startdef infourthree = simplify_polynomial('x',[cfourthree,0]) enddef
startdef outfourthree = simplify_polynomial('y',[0,dfourthree,0]) enddef

startall
startquestion  $\infourthree \outfourthree$ endquestion
startsolution \item  endsolution
endall

Note that no sign needs to be written before \outfourthree, as a result of this construction.



%%4.4 - Numbers with commas

This call can be used to write numbers with comma separators, for ease of use.

startdef afourfour = np.random.randint(1000000,2000000) enddef
startdef nicefourfour = simplify_commanumber(afourfour) enddef

Note that LaTeX treats commas in math mode like proper commas, so the display is weird if used in math mode. Outside of math mode, you must manually insert a space after the number (if so desired) to have it work correctly in a sentence. 

startall
startquestion $\afourfour$ is awkward to read; \nicefourfour is missing a space, \nicefourfour\ is nice, $\nicefourfour$ adds spacing that makes it confusing to read. endquestion
startsolution \item  endsolution
endall

This does not work correctly with decimal floats currently, in that trailing zeros will be lost. The decimals themselves work correctly.

startdef goodfourfour = simplify_commanumber(1000000.12345) enddef
startdef badfourfour = simplify_commanumber(1000000.10000) enddef

startall
startquestion  \goodfourfour $\neq$ \badfourfour endquestion
startsolution \item  endsolution
endall
