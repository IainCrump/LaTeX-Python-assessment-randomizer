This file discusses how to create multiple choice questions. In general, any type of randomization that can be done for long-answer questions can be done for multiple choice, and as such will not be discussed here.

Again, to create a multiple choice question, the file name must start with the letter 'm'.


As there are fewer things that can be done with multiple choice, the layout here will be less organized.



startpre
\item {\bf (2 points)} 
endpre

startpost
endpost



Key to multiple choice is the "solset" bookends. Three variables exist inside;
-  "column" bookends control how many columns the possible answers are split into
-  "width" bookends controls the textwidth fraction of the page the answers occupy
-  "sola -- solb bookends are the answers.

The first listed answer is what is selected as the correct answer for the solution file.

Any number of answers can be presented.


startall
startquestion Which of these is correct?
endquestion

startsolset
begincolumn 3 endcolumn
beginwidth 1.0 endwidth

sola Correct. solb
sola Wrong. solb
sola Wrong. solb
endsolset
endall




startall
startquestion Which of these is correct?
endquestion

startsolset
begincolumn 4 endcolumn
beginwidth 1.0 endwidth

sola Correct. solb
sola Wrong. solb
sola Wrong. solb
sola Wrong. solb
sola Wrong. solb
sola Wrong. solb
sola Wrong. solb
sola Wrong. solb
sola Wrong. solb
sola Wrong. solb
endsolset
endall



If a solution bookend contains the phrase "always_first", it will always be the first listed answer. Similarly, if it contains the string "always_last", it will always be the last listed answer.

These can be omitted, as with the previous problem, or used individually or together

Placement within the bookends is unimportant. The string is found and removed, so it can be placed in the middle of a word without affecting the appearance of the answer.

startall
startquestion Which of these isn't mentally problematic?
endquestion

startsolset
begincolumn 1 endcolumn
beginwidth 1.0 endwidth

sola None of the below. always_first solb
sola $a \neq a$ solb
sola I've built a set that contains itself. solb
sola always_last All of the above. solb
endsolset
endall

